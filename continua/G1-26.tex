\documentclass[11pt,a4paper]{article}

%-------------------------------------------
% Paquetes básicos
%-------------------------------------------
\usepackage[spanish]{babel}
\usepackage[utf8]{inputenc}
\usepackage[T1]{fontenc}
\usepackage{lmodern}
\usepackage{amsmath, amssymb, amsthm}
\usepackage{geometry}
\usepackage{enumitem}
\usepackage{hyperref}

\geometry{margin=2.5cm}

%-------------------------------------------
% Entorno para ejercicios
%-------------------------------------------
\newtheoremstyle{ejercicio}
  {10pt}   % espacio arriba
  {10pt}   % espacio abajo
  {\normalfont}
  {}
  {\bfseries}
  {.}
  {0.5em}
  {}

\theoremstyle{ejercicio}
\newtheorem{ejercicio}{Ejercicio}

%-------------------------------------------
\title{Guía de Ejercicio I:\\Oda a la Interpolación}
\author{nICO no IKO 777}

\date{}

%-------------------------------------------
\begin{document}

\maketitle
\textit{El primer principio es que no debes engañarte a ti mismo y eres la persona más fácil de engañar. Una vez que no te engañas a ti mismo, es fácil que no engañes a los otros científicos.}  R.  Feynman.

%%%%%%%%%%%
\section*{Indicaciones para el desarrollo y la entrega}

\begin{itemize}

\item \textbf{Formato de entrega.}  
La entrega debe realizarse exclusivamente en formato PDF,
generado a partir de un documento escrito en \LaTeX.
No se aceptarán entregas en formato Word u otros editores.

\item \textbf{Presentación.}  
Sea ordenado y claro en la redacción.  
Incluya demostraciones completas en los ejercicios teóricos
y explique adecuadamente los resultados obtenidos en los ejercicios prácticos.
Las gráficas deben estar correctamente rotuladas y comentadas.

\item \textbf{Plazo de entrega.}  
El plazo es de \textbf{dos semanas contadas a partir del 20/02/2026}.  
La fecha límite de entrega es el \textbf{06/03/2026}. 

\item \textbf{Uso de consultas.}  
Pueden realizar consultas durante el desarrollo de la guía.
Con todo gusto los orientaré.

\item \textbf{Clase de apoyo.}  
El martes 24 utilizaremos la clase para resolver ejercicios
y discutir dificultades comunes.  
Si es necesario, podremos organizar una clase adicional optativa
para profundizar en los temas más complejos.

\item \textbf{Uso de Inteligencia Artificial.}  
Se permite el uso de herramientas de IA como apoyo.
Sin embargo, deben ser plenamente conscientes de lo que están haciendo.
La IA puede cometer errores o generar argumentos incorrectos.
Usted es responsable del contenido que entrega.

\end{itemize}

%%%%%%%%%%%

\section*{Parte I -- Ejercicios Teóricos}
%[Unicidad del polinomio interpolante]
\begin{ejercicio}
Sea $\{(x_i,y_i)\}_{i=0}^n$ un conjunto de nodos con $x_i$ distintos.

\begin{enumerate}[label=\alph*)]
\item Demuestre que existe un único polinomio $p \in \mathbb{P}_n$ tal que
\[
p(x_i)=y_i, \quad i=0,\dots,n.
\]

Sugerencia: Suponga que existen dos polinomios $p$ y $q$
que interpolan los mismos datos y estudie el polinomio $r=p-q$.

\item Concluya utilizando el hecho de que un polinomio no nulo
de grado $\le n$ no puede tener más de $n$ raíces.
\end{enumerate}
\end{ejercicio}

%-------------------------------------------
%[Fórmula del error de interpolación]
\begin{ejercicio}
Sea $f \in C^{n+1}([a,b])$ y sea $p_n$ su polinomio interpolante
en nodos $x_0,\dots,x_n$.

\begin{enumerate}[label=\alph*)]
\item Demuestre que el error puede escribirse como
\[
f(x)-p_n(x)=
\frac{f^{(n+1)}(\xi_x)}{(n+1)!}
\prod_{i=0}^n (x-x_i)
\]
para algún $\xi_x \in (a,b)$.

\item Indique en qué parte del argumento se utiliza el teorema de Rolle.

\item Discuta qué ocurre si los nodos están muy próximos entre sí.
\end{enumerate}
\end{ejercicio}


%[Los polinomios de Lagrange forman una base]
\begin{ejercicio}

Sean $x_0,\dots,x_n$ nodos distintos y sean los polinomios fundamentales
de Lagrange definidos por

\[
\ell_i(x)=
\prod_{\substack{0 \le j \le n \\ j \ne i}}
\frac{x-x_j}{x_i-x_j},
\quad i=0,\dots,n.
\]

\begin{enumerate}[label=\alph*)]

\item Demuestre que
\[
\ell_i(x_j)=\delta_{ij},
\]
donde $\delta_{ij}$ es el delta de Kronecker.

\item Pruebe que el conjunto $\{\ell_0,\dots,\ell_n\}$ es linealmente independiente.

\item Concluya que $\{\ell_0,\dots,\ell_n\}$ forma una base del espacio
$\mathbb{P}_n$ de los polinomios de grado menor o igual que $n$.

\item Compare esta base con la base canónica
\[
\{1,x,x^2,\dots,x^n\}.
\]
¿Qué ventaja tiene la base de Lagrange en el contexto de interpolación?

\end{enumerate}

\end{ejercicio}


%-------------------------------------------
\section*{Parte II -- Ejercicios Prácticos}
% [Interpolación de Newton con datos reales]
\begin{ejercicio}
Implementar la interpolación de Newton a partir de datos experimentales.

\begin{enumerate}[label=\alph*)]
\item Cargar un archivo CSV con dos columnas numéricas.
\item Agrupar los datos si existen nodos repetidos.
\item Construir la tabla de diferencias divididas.
\item Evaluar el polinomio interpolante.
\item Dibujar el polinomio junto con los datos originales.
\item Construir un spline cúbico con los mismos datos.
\item ¿Cuál método recomendaría en la práctica? Justifique.
\item Cosntruya una situacion donde a los datos originales, le agregas, nuevos datos e interpolas.
\end{enumerate}

%\textbf{Extensión:} Escalar la variable independiente y comparar resultados.
\end{ejercicio}

%-------------------------------------------
% [Sensibilidad frente a perturbaciones]
\begin{ejercicio}
Estudiar la sensibilidad del polinomio interpolante.

\begin{enumerate}[label=\alph*)]
\item Perturbar un solo dato $y_k$ en una cantidad pequeña $\delta$.
\item Construir el nuevo polinomio interpolante.
\item Dibujar la diferencia
\[
|p(x)-\tilde{p}(x)|.
\]
\item Comparar la magnitud de la perturbación con el error máximo obtenido.
%\item Concluir si el problema es bien o mal condicionado.
\end{enumerate}
\end{ejercicio}

%-------------------------------------------
% [Aproximantes de Padé y comparación con interpolación]
\begin{ejercicio}
Aproximantes de Padé.\\
Sea $f$ una función analítica en un entorno de $x=0$, con desarrollo de Taylor

\[
f(x)=\sum_{k=0}^{\infty} a_k x^k.
\]

\begin{enumerate}[label=\alph*)]

\item \textbf{Definición.}
Defina el aproximante de Padé $[m/n]_f(x)$ como una función racional

\[
R_{m,n}(x)=\frac{P_m(x)}{Q_n(x)},
\]

donde $P_m$ y $Q_n$ son polinomios de grado $\le m$ y $\le n$
respectivamente, con $Q_n(0)=1$, tal que

\[
f(x)-R_{m,n}(x)=\mathcal{O}(x^{m+n+1})
\quad \text{cuando } x \to 0.
\]

\begin{itemize}
\item Entienda la definición con un ejemplo
\item Explique en qué sentido este aproximante ``coincide'' con la serie de Taylor.
\end{itemize}
\bigskip

\item \textbf{Construcción algorítmica.}
A partir de los coeficientes $a_0,\dots,a_{m+n}$,
deduzca el sistema lineal que permite calcular los coeficientes
de $Q_n(x)$ y posteriormente los de $P_m(x)$.

Implemente un algoritmo en Python o Matlab que:

\begin{itemize}
\item Entienda el problema con un ejemplo (Todo lo que sigue a continuación, lo puede hacer con el ejemplo)
\item reciba como entrada los coeficientes de Taylor, 
\item construya el aproximante $[m/n]$,
\item evalúe la función racional obtenida.
\end{itemize}

\bigskip

\item \textbf{Comparación con interpolación polinómica.}

Considere la función
\[
f(x)=e^x \quad \text{o} \quad f(x)=\frac{1}{1+x^2}.
\]

\begin{enumerate}
\item Construya el polinomio de Taylor de orden $m+n$. (Puede hacerlo con un ejemplo)
\item Construya el aproximante de Padé $[m/n]$.
\item Dibuje ambos aproximantes en un intervalo adecuado.
\item Compare visualmente el comportamiento cerca del origen y lejos de él.
\end{enumerate}

\bigskip

\item \textbf{Discusión del error.}

\begin{enumerate}
\item Analice el error absoluto
\[
|f(x)-R_{m,n}(x)|
\quad \text{y} \quad
|f(x)-T_{m+n}(x)|.
\]

\item ¿En qué regiones el aproximante de Padé presenta menor error?

\item Discuta por qué los aproximantes racionales pueden capturar mejor
comportamientos asintóticos o singularidades que los polinomios.

\item Relacione este comportamiento con el fenómeno de Runge
y la estabilidad numérica.
\end{enumerate}

\end{enumerate}

\end{ejercicio}




\end{document}
